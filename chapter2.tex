\section{问题分析}
\subsection{针对问题一的分析}

问题1的处理首先要建立在数据的高质量预处理上,将问题中给出的四类测试基本信息选出合适的相应指标作为特征值进行相关性分析或训练。分析其对AP发送机会的影响,若参数对AP发送机会重要性强,那么AP 发送受其影响则更大,则AP发送数据帧序列总时长(seq\_time)将会变短;反之参数若对AP发送机会干扰弱,那么AP发送的概率或次数将增大增多,则AP发送数据帧序列总时长将会边长。

给出的数据中RSSI数值数量较多且不宜直接应用,应先对RSSI进行处理,并通过题目中给出的关系,尽可能构建对AP发送数据帧序列总时长相关的新特征,可以更好的在后续的模型训练中得到更大的贡献值。对所有特征值指标对seq\_time指标完成相关性分析后,可以通过机器学习模型训练反过来排列对seq\_time指标影响的重要度排名对相关性分析进行一定的校准与验证,最后利用相关度较高的特征,对seq\_time指标进行预测模型的训练并完成测试集的预测任务。

\subsection{针对问题二的分析}
由于不同数据处在不同数量的BSS下,其可被影响的AP 数量也不同,在需要预测的数据中也是不同个数AP数据,故可按照同频AP个数分类建模。但对不同AP个数进行统一建模也可以进行训练与预测完成问题的任务,所以为了更好地确认分类建模效果好还是统一建模效果好团队预将三个模型训练效果进行对比,以确保模型地高精度。

\subsection{针对问题三的分析}