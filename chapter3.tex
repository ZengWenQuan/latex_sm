\section{模型假设与符号说明}
\subsection{模型的假设}
\begin{assumption}
	\label{asm:1}
	数据中包括的三个AP均处在同一信道中,即属于同频AP,节点之间发送数据会造成相互干扰与影响且属于同频干扰;
\end{assumption}

\begin{assumption}
	\label{asm:2}
	假设在场景中AP或STA接受数据时收到的环境底噪较小,忽略不计,干扰均来自于其他AP或STA传输的数据;
\end{assumption}

\begin{assumption}
	\label{asm:3}
	假设TCP协议在本题中不考虑三次握手和四次挥手建立连接过程,即只考虑数据传输过程;
\end{assumption}

\begin{assumption}
	\label{asm:4}
	假设在场景中只存在同频干扰,不存在异频干扰,且相邻AP不进行通信。
\end{assumption}

\subsection{主要符号说明}
\subsubsection{缩略字母说明,其他缩略字母请见正文部分}

\begin{table}[H]
	\centering
	\caption{缩略字母说明}
	\begin{tabular}{p{2.0cm}<{\centering}p{8.0cm}<{\centering}p{4.0cm}<{\centering}}
		%指定单元格宽度, 并且水平居中。
		\hline
		缩略字母 & 全称 & 中文含义  \\ %换行 
		\hline
		$AP$ & Access Point & 无线接入点   \\ %把你的符号写在这
		STA	& Station	& 站点\\
		
		RSSI &	Received Signal Strength Indication	& 接收信号能量强度\\
		
		WLAN &	Wireless Local Area Network	& 无线局域网\\
		
		CCA &	Clear Channel Assessment	& 信道可用评估\\
		
		PD &	Packet Detection	& 包检测门限\\
		
		ED &	Energy Detection	& 能量检测门限\\
		
		NAV	 & Network Allocator Vector	& 网络分配矢量\\
		
		SINR &	Signal To Interference And Noise Ratio	& 信干噪比\\
		
		MCS &	Modulation and Coding Scheme	& 调制编码方案\\
		
		NSS &	Number of Spatial Stream	& 空间流数\\
		
		BSS &	Basic Service Set	& 基本服务集\\
		
		\hline
	\end{tabular}
\end{table}

\subsubsection{参数符号说明}
\begin{longtable}{ c c c }
	\caption{参数符号说明} \\ % 表格标题
	%指定单元格宽度, 并且水平居中。
	\hline
	符号 & 含义   \\ %换行 
	\hline
	
	$d$	& 接收信号的距离\\
	$d_0$ &	接收信号的及基准距离\\
	$ \lambda $	& 路径动态衰减参数\\
	$X_{\theta}$ &	均值为$\sigma$标准差为$\theta$的高斯分布随机函数\\
	$P_s$ &	数据信号的功率值\\
	$P_i$ &	干扰信号的功率值\\
	$P_n$ &	环境底噪的功率值\\
	$RSSI_s$ &	数据信号的接收信号强度\\
	$RSSI_i$ &	干扰信号的接收信号强度\\
	$NAV_i$ &	当前信道i的NAV门限值\\
	$v_i$ &	第i个数据的剩余误差\\
	$ \alpha $ &	同步传输强度占混合传输强度的比例\\
	$\beta_1$ &	信号强度处于PD和ED门限之间的信号中携带preamble前导头的概率\\
	$\beta_2$ &	信号强度超过ED门限的信号中能被准确识别的判断信道繁忙的概率\\
	$C_{HN}$ &	奈奎斯特信道传输速率\\
	$C_{CS}$ &	香农信道传输速率\\
	
	\hline
\end{longtable}