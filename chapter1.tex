\section{问题重述}
\subsection{问题背景}   % 问题的背景

无线局域网(WLAN)已成为现代通信基础设施不可或缺的一部分,尤其是在需要灵活移动连接的场景下。在密集部署的环境中,同频组网作为一种实现零漫游的组网方式,被广泛应用。在这种配置下,各个AP之间的干扰成为了一个显著的问题,特别是同频干扰或共信道干扰(CCI),它会导致数据包丢失、重传增加、网络覆盖范围缩减以及连接不稳定等问题,从而严重影响了WLAN的性能和服务质量。

然而,传统的基于仿真的模型在实际部署中往往无法达到预期的效果,因为现实世界中的WLAN通信面临着信道条件快速变化、多种干扰源以及复杂的服务流量等问题。因此,本研究致力于基于实际测量数据,深入分析同频AP环境下的各种影响因素,如网络拓扑、RSSI、信道接入机制及同频干扰等,从而实现更为精确的吞吐量预测。通过这种方式,我们期望能够提出一种更贴近实际应用需求的优化方案,以改善WLAN系统的性能表现。

本研究旨在基于WLAN的同频组网实测数据,分析网络拓扑、节点间RSSI、信道接入机制、干扰等因素对数据传输速率的影响,进而实现对WLAN系统吞吐量的精确预测。通过该预测模型对WLAN系统进行优化,有望在工业、教育、医疗等新兴场景中实现突破,为用户提供卓越的业务体验。

\subsection{问题描述}
\subsubsection{问题一}
首先根据附件WLAN网络实测训练集中所提供的网络拓扑、业务流量、门限、节点间RSSI的四类测试基本信息,分析其中各参数对AP发送机会的影响,并给出影响性强弱的顺序。并且通过训练的模型,预测每个AP的发送机会或是概率,即发送数据帧序列的总时长(seq\_time),并通过测试集 test\_set\_1\_2ap和test\_set\_1\_3ap预测AP发送数据帧序列的总时长。可按照同频AP个数分类分析和分别建模,也可统一分析和建模。

其中包含三个任务:第一个任务为利用训练集中提供的四类测试基本信息对AP发送机会的影响强弱进行判断并排序。第二个任务为利用训练集进行训练,预测AP发送的机会,即发送数据帧序列总时长;第三个任务为利用训练好的模型对两个测试集进行AP发送数据帧序列的总时长。

为了分析WLAN网络中各参数对AP发送机会的影响并建立相应的预测模型,我们首先依据提供的网络拓扑、业务流量、门限及节点间RSSI这四类基本信息,评估这些参数对AP发送机会的具体影响,并确定各个因素影响的强弱顺序;随后,使用训练集中的数据进行模型训练,以预测每个AP的发送机会,具体表现为发送数据帧序列的总时长(seq\_time);最后,运用训练好的模型对两个特定的测试集(test\_set\_1\_2ap 和 test\_set\_1\_3ap)进行预测,确定在不同数量的同频AP环境下AP发送数据帧序列的总时长。整个过程既可以通过按同频AP的数量分类分析和分别建模的方式进行,也可以采取统一分析与建模的方法,以确保模型能够准确反映实际情况并提供可靠的预测结果。

\subsubsection{问题二}
根据附件提供的实测训练集中的问题1中提到的四类测试基本信息,特别是节点间RSSI信息和门限信息,结合问题1中对AP发送机会的分析,对测试中AP发送数据选用最多次数的(MCS, NSS)进行建模,并通过测试集 test\_set\_2\_2ap和test\_set\_2\_3ap预测(MCS, NSS)。

AP在自适应调制与编码AMC算法下自适应调节发送速率,AP通过监测信道条件SINR信干燥比的大小,在过程中动态地采用多个(MCS, NSS),为了使AMC算法收敛速度快,故其中选用最多次数的(MCS, NSS)代表了此组合下是当前信道的最佳组合。

在WLAN网络中,AP通过自适应调制与编码(AMC)算法根据信道条件(如SINR)动态调整发送速率,选择使用次数最多的(MCS, NSS)组合作为当前信道的最佳组合,要求根据训练集中的网络拓扑、业务流量、门限信息、特别是节点间RSSI信息,分析这些参数对AP发送机会的影响,找出与MCS和NSS相关性较高的特征值,并基于这些特征值训练模型预测在给定信道条件下AP最常选用的(MCS, NSS)组合,最终使用训练好的模型对两个测试集(test\_set\_2\_2ap 和 test\_set\_2\_3ap)进行预测,得到AP最常选用的(MCS, NSS)组合。

\subsubsection{问题三}
结合问题1和问题2的分析,对系统吞吐量进行建模,并通过测试集test\_set\_1\_2ap和test\_set\_1\_3ap预测网络吞吐量。无线信道具有瞬息万变的特点,实测中所测量的RSSI信息属于大尺度信息,不足以完全反应真实信道变化,因而问题2对(MCS, NSS)的建模可能无法获得很高精度。允许采用实测中统计的数据帧真实(MCS, NSS)作为模型输入变量。在题目的基础上,不仅需要结合前两文的内容,对于问题2的MCS、NSS放入模型,还要从题干中提取新的特征值完成模型的训练。
